We now know enough to begin proving Steinitz's Theorem. First let's
recall the statement of \cref{thm:steinitz}

\steinitz*

Next we recall the following two results from \cref{sec:Algebraic Background}.

\algextend*

\transextend*

The first result allows us to create algebraic extensions, the next one allows
us to create transcendental extensions. Combining these two results with the
back and forth method and transfinite induction is going to enable us to give
a proof of \cref{thm:steinitz}

\begin{proof}[Proof of \cref{thm:steinitz}]
  Suppose that $M$ and $N$ are algebraically closed fields of characteristic
  0 such that $|M| = |N| = \kappa > \aleph_{0}$. We enumerate $M$ and $N$ as
  \[
  M = \{a_{\alpha}: \alpha < \kappa\} \qquad and \qquad N = \{b_{\alpha}:
  \alpha < \kappa\} 
  .\] 

  We now wish to construct a sequence of field extensions $E_0 \subseteq E_1
  \subseteq E_{\alpha} \subseteq \ldots \subseteq M$ and $K_0 \subseteq K_1
  \subseteq \ldots \subseteq K_{\alpha} \subseteq N$ such that $|E_{\alpha}|,
  |K_{\alpha}| < \kappa$
  as well as a sequence of
  functions $f_0 \subseteq f_1 \subseteq \ldots f_{\alpha} \subseteq \ldots$
  such that $f_{\alpha}: E_{\alpha} \to K_{\alpha}$ is a partial isomorphism.
  Since $M$ and $N$ are of characteristic zero we know that both of their prime
  fields are $\mathbb{Q}$. This gives us our base case $f_0: \mathbb{Q} \to
  \mathbb{Q}$ which is given by the identity map (as there are no non-trivial
  automorphisms of $\mathbb{Q}$).

  Next we handle the successor case. Let $\beta = \alpha + 1$.
  Suppose that we have constructed subfields
  $E_{\alpha}$ and $K_{\alpha}$ as well as an isomorphism $f_{\alpha}:
  E_{\alpha} \to K_{\alpha}$. We are going to do both the back and the forth
  part in one step. First suppose that $b_{\beta} \in N \setminus K_{\alpha}$ then by
  either \cref{propn:algextend} or \cref{propn:existtrans} there is an $a \in
  M \setminus E_{\alpha}$ and an extension $f'_{\alpha}: E_{\alpha}(a) \to
  K_{\alpha}(b)$. Now suppose that $c \in M \setminus E_{\alpha}(a)$ then once
  again by the appropriate proposition there is a $d \in N \setminus
  K_{\alpha}(b)$ and an extension $f''_{\alpha}: E_{\alpha}(c) \to
  E_{\alpha}(d)$. Thus we can take $f_{\beta} = f''$ and $E_{\beta}
  = E_{\alpha}(c)$ and $K_{\beta} = K_{\alpha}(d)$. As we needed.

  Lastly in the limit case we can take $f_{\alpha} = \bigcup_{\beta < \alpha}
  f_{\beta}$, $E_{\alpha} = \bigcup_{\beta < \alpha} E_{\beta}$ and $K_{\alpha}
  = \bigcup_{\beta < \alpha} K_{\beta}$.

  Therefore using transfinite induction we can define this recursively for all
  ordinals (including $\kappa$). The resulting function is an isomorphism as
  a union of partial isomorphisms. As we needed.
\end{proof}

