We begin with some of the relevant algebraic background which is needed to
understand both the statment of the problem and the proof. First recall the
definition of a field:

\begin{propn}
  Every field homomorphism is injective 
\end{propn}
\begin{proof}
  Suppose that $F, K$ are fields and let $\phi: F \to K$ a field homomorphism
  between them. Let $a, b \in F$ and suppose that $\phi(a) = \phi(b)$. This
  implies that $\phi(a) - \phi(b) = 0$ so $\phi(a - b) = 0$ if $a \neq b$ then
  this means that $\phi(a-b) = 0$ for $a - b \neq 0$. Taking $c = a-b$ then we
  know that $\phi(cc^{-1}) = 1 = \phi(c)\phi(c^{-1}) = 0$ which is
  a contradiction. Thus $a = b$ necessarily and so $\phi$ is injective.
\end{proof}

This fact that field homomorphisms are always injective lies at the heart of
field theory, in particular it is the reason why we are able to focus so
heavily on what are known as field extensions. 

\begin{defn}[Field Extension]
  We say that $E$ is a field extension of $F$ if there is an injective field
  homomorphism $\phi: F \to E$
\end{defn}

Field extensions come in two main flavours: algebraic extensions and
trancendental extensions. We focus on the former first.

\begin{defn}[Polynomial Ring over $F$]
  Suppose that $F$ is a field we define the polynomial ring  $F[x]$ over $F$ as
  \[
    F[x] := \left\{\sum_{i=1}^{n}a_{i}x^{i}: a_{i} \in F, n \in
    \mathbb{N}\right\} 
  .\]
  We take the operations to be the usual polynomial addition and product.
\end{defn}

\begin{defn}[Irreducible]
  We say that a polynomial $p(x) \in F[x]$ is irreducible if it cannot be
  factored into two polynomials of lesser degere.
\end{defn}

This notion of irreducibility is one that we are already very familiar with.
Consider the example of $x^2 + 1$ as a polynomial over $\mathbb{Q}$. Clearly
this polynomial has no factors over $\mathbb{Q}$ (in particular since $i
\not\in \mathbb{Q}$) and so as a result we know
that we cannot factor this polynomial any further. A similar example is that of
$x^2 - 2$. This polynomial taken over $\mathbb{R}$ has a root of $\sqrt{2}$
however $\sqrt{2}$ is famously not a rational number. Something we might wish
to do now is ``extend'' $\mathbb{Q}$ in order to contain these elments so that
these polynomials factor over them.

\begin{defn}[Algebraic Numbers]
  Let $F$ be a field and $E$ a field extension of $F$.
  We say that $\alpha \in E$ is algebraic over a field $F$ if there is a 
  polynomial $p(x) \in F[x]$ such that $p(\alpha) = 0$ when viewed as
  a polynomial over $E$.
\end{defn}

Without going into too many details about how these extensions work we are
going to loosely define what it means to be an algebraic extension of $F$. Just
know that this is a process that can be done rigorously for any irreducible
polynomial in $F[x]$. The smallest (in degree) irreducible polynomial
containing $\alpha$ as a root is known as the minimal polynomial of $\alpha$
and is denoted $m_{\alpha}$.

\begin{defn}[Algebraic Extension]
  Suppose that $\alpha$ is algebraic over $F$. We define $F(\alpha)$ as the
  smallest field which contains $F$ and $\alpha$.
\end{defn}

We know that algebraic numbers satsify some polynomial $p(x)$ over $F$, 
so we can think of the field $F(\alpha)$ as adding in the root of this
polynomial. One important fact that we are going to make heavy use of later is
that we are able to lift isomorphisms of fields up to isomorphisms of their
extensions. In the case of algebraic extensions this is relatively
straightforward since we know ``exactly'' what algebraic numbers look like in
terms of the ring of polynomials. 

We first notice that we can lift isomorphisms between fields to an isomorphism
between their polynomial rings. 

\begin{lemma}\label{lemma:extendtoring}
  Let $F$ and $K$ be fields and if $\phi: F \to K$ be an isomorphism between
  them then there exists a (ring-)isomorphism $\overline{\phi}: F[x] \to
  K[x]$. Moreover, if $f(x)$ is irreducible over $F$ then $\overline{\phi}(f)$ 
  is irreducible over $K$.
\end{lemma}
%\begin{proof}
%  We define $\overline{\phi}: F[x] \to K[x]$ by
%  \[
%  f(x) = \sum_{i=1}^{n} a_{i}x^{i} \mapsto \sum_{i=1}^{n} \phi(a_{i})x^{i}
%  .\] 
%  This is an isomorphism because $\phi$ is an isomorphism (details left as an
%  exercise to the reader). Now suppose that $f$ is an irreducible polynomial
%  over $F$ and suppose that $\overline{\phi}(f)$ is not irreducible. This means
%  that there are polynomials $a(x), b(x)$ such that $\overline{\phi}(f)(x)
%  = a(x)b(x)$. However this means that
%  \[
%  f = \overline{\phi}^{-1}(\overline{\phi}(f)) = \overline{\phi}^{-1}(a(x)b(x))
%  = \overline{\phi}^{-1}(a(x))\overline{\phi}^{-1}(b(x))
%  .\] 
%  This implies that one of $\overline{\phi}^{-1}(a(x)),
%  \overline{\phi}^{-1}(b(x))$ must be the identity map and so one of $a(x)$ or
%  $b(x)$ is identity. Thus $\overline{\phi}(f)$ is irreducible as needed.
%\end{proof}


\begin{restatable}{propn}{algextend}
\label{propn:algextend}
  If $F$ and $K$ are fields and there is an isomorphism $\phi: F \to K$ and
  $\alpha$ is algebraic over $F$ then there is a $\beta$ algebraic over $K$ 
  such that $\phi$ can be extended to an isomorphism $\overline{\phi}:
  F(\alpha) \to K(\beta)$
\end{restatable}
\begin{proof}
  If $\alpha$ is algebraic over $F$ then we know that $\alpha$ is the root of
  some minimal polynomial $m_{\alpha}(x) = x^{n} + a_{n-1}x^{n-1} + \ldots
  + a_1x + a_0 \in F[x]$. Now since $\phi$ is an isomorphism of fields we get
  an irreducible polynomial 
  \[
    \overline{m_{\alpha}}(x) = x^{n} + \phi(a_{n-1})x^{n-1} + \ldots
    + \phi(a_1)x + a_0 \in K[x]
  .\] 
  Now by \cref{lemma:extendtoring} we get an isomorphism $\overline{\phi}: F[x]
  \to K[x]$ we notice that under this map the ideal $(m_{\alpha})$ is mapped to
  $(\overline{m_{\alpha}})$. Lastly we note that $F[x]/(m_{\alpha})$ is
  isomorphic to $F(\alpha)$ and $K[x]/(\overline{m_{\alpha})}$ is isomorphic to
  $K[\beta]$ by definition, label these isomorphisms $\psi_{\alpha}$ and
  $\psi_{\beta}$ respectively. This gives the following diagram:
%  \begin{equation}
%  \begin{tikzcd}
%    F \arrow[r, "\cong"] \arrow[d, hook]& K \arrow[d, hook] \\ 
%    F[x] \arrow[r,"\cong"] \arrow[d, "\pi"] & K[x] \arrow[d,"\pi"] \\
%    F[x]/(m_{\alpha}) \arrow[r, "\sigma"] \arrow[d, "\psi_{\alpha}"] & 
%                                          K[x]/(\overline{m_{\alpha}})
%                                          \arrow[d,"\psi_{\beta}"]\\
%    F(\alpha) \arrow[r] & K(\beta) \\
%  \end{tikzcd}
%  \end{equation}
  \begin{equation}
  \begin{tikzcd}
    F  \arrow[r,hook] \arrow[d,"\phi"] & F[x] \arrow[r] \arrow[d,"\overline{\phi}"] 
                                        & F[x]/(m_{\alpha})
    \arrow[r,"\psi_{\alpha}"] \arrow[d,"\sigma"] & F(\alpha) \arrow[d,dashed] \\
    K \arrow[r,hook] & K[x] \arrow[r] & K[x]/(\overline{m_{\alpha}})
    \arrow[r,"\psi_{\beta}"] & K(\beta)
  \end{tikzcd}
  \end{equation}
  
  We notice that the map $\sigma: F[x]/(m_{\alpha}) \to
  K[x]/(\overline{m_{\alpha}})$ is an isomorphism since $\overline{\phi}$ is an
  isomorphism. Now reading off the diagram we get that the map
  $\psi_{\beta}g\psi_{\alpha}^{-1}$ is the isomorphism we require.
\end{proof}

Moving on from algebraic numbers we are now interested in exploring the second
class, trancendental numbers.  These are all numbers which are not the root of 
any polynomial over the ground field. 

\begin{defn}[Trancendental Number]
  Suppose that $E$ is some field extension of $F$ and let $\alpha \in E$. We
  say that $\alpha$ is trancendental over $F$ if it is not algebraic over $F$.
\end{defn}

\begin{defn}[Trancendental Extension]
  Similar to the case of algebraic extensions suppose that $\alpha$ is
  trancendental over $F$ then we define $F(\alpha)$ as the smallest field
  containing both $\alpha$ and $F$.
\end{defn}

Recall that our goal is to prove an analogous result to \cref{propn:algextend}
for trancendental numbers / extensions. In order to do this we are first going
to need to prove a couple of facts. 

\begin{propn}\label{propn:existtrans}
  Let $F$ be a field. There are at most $|F|$ many algebraic elements over $F$.
\end{propn}
\begin{proof}
  This is just a counting problem. Each algebraic number $\alpha$ over $F$ is
  the root of some polynomial $x^{n} + \ldots + a_1x + a_0$. There are at most
  countably many polynomials in which we have $|F|$ many chocies for
  coefficients. Thus there are at most $|F|$ many algebraic numbers.
\end{proof}

This proposition lets us make sense of why there \textit{must} be trancendental
numbers in $\mathbb{R}$ when viewed as an extension of $\mathbb{Q}$. This is
because there are uncountably infinitely many real numbers and only countably
many algebraic numbers over $\mathbb{Q}$. 

The next tool that we are going to need is the quotient field / field of
fractions associated with the polynomial ring $F[x]$. The details about this
space can be found in \cite{Dummit_Foote_2004} but for us we are going to
strictly think about them as formal expresstions.

\begin{defn}[Rational Polynomials]
  Let $F$ be a field, we define $F(x)$ as
  \[
    F(x) := \left\{\frac{p(x)}{q(x)}: p(x), q(x) \in F[x]\right\} 
  .\] 
  This is known as the rational polynomials over $F[x]$.
\end{defn}

\begin{lemma}\label{lemma:isotoquot}
  If $F$ is a field and $\beta$ is trancendental over $F$ then $F(x) \cong
  F(\beta)$ 
\end{lemma}
\begin{proof}
  We define the map $ev_{\beta}: F(x) \to F(\beta)$ which is given by 
\[
  \frac{p(x)}{q(x)} \mapsto \frac{p(\beta)}{q(\beta)}
  .\] 
  First note that if $\beta$ was not trancendental then this map is not injective
  (or well defined) as we end up dividing by 0 whenever $m_{\alpha}(x) \mid
  q(x)$. However since $\beta$ is not trancendental, this map is not only
  injective, but it is infact an isomorphism! Since $F(x)$ is a field by
  construction and $F(\beta)$ is also a field. Thus $ev_{\beta}$ is certainly
  surjective. Moreover, it is also an isomorphism as it is also clearly
\end{proof}

Lastly we are going to need a lemma that allows us to lift isomorphisms between
fields to isomorphisms between their rational polynomials

\begin{lemma}\label{lemma:lifttoquot}
 Suppose that $F$ and $K$ are fields and $\phi: F \to K$ is an isomorphism.
 Then $\phi$ can be lifted to an isomorphism $\overline{\phi}: F(x) \to K(x)$. 
\end{lemma}
\begin{proof}
  This follows immediately from the fact that $\phi$ is an isomorphism. We can
  simply apply $\phi$ to the coefficients of the coefficients of any
  polynomials. Since equality is done coefficient wise this gives our result.
\end{proof}

Now knowing this we can prove our main result for trancendental numbers.

\begin{restatable}{propn}{transextend}
  Let $M$ and $N$ be field extensions of $F$ and $K$ respectively.
  If $\phi: F \to K$ is an isomorphism and $\alpha \in M \setminus F$ 
  is trancendental over $F$ then there is $\beta \in N \setminus K$ which is
  trancendental over $K$ such that $\phi$ can be extended to an isomorphism 
  $\overline{\phi}: F(\alpha) \to K(\beta)$
\end{restatable}
\begin{proof}
  Since we know that $|K| < |N|$ by \cref{propn:existtrans} there is an
  element $\beta$ which is trancendental over $K$. Now by
  \cref{lemma:isotoquot} we have the following:
  \[
    F \xhookrightarrow{} F(x) \cong F(\alpha)
  \] 
  and
  \[
    K \xhookrightarrow{} K(x) \cong K(\beta)
  .\] 
  Labelling the isomorphism between $F(x)$ and $F(\alpha)$ as $\psi_{\alpha}$ 
  and the isomorphism between $K(x)$ and $K(\beta)$ as $\psi_{\beta}$.
  Additionally by \cref{lemma:lifttoquot} there is an isomorphism $\phi':
  F(x) \to K(x)$ which lifts $\phi$ to an isomorphism between the rational
  polynomials. Fromt these we get the following diagram:
  \begin{equation}
  \begin{tikzcd}
    F  \arrow[rr] \arrow[d, hook] & & K \arrow[d, hook] \\
    F(x) \arrow[rr] \arrow[d, "\psi_{\alpha}"']& & K(x) 
                                        \arrow[d, "\psi_{\beta}"]
    \\
    F(\alpha) \arrow[rr, dashed, "\psi_{\beta}\phi'\psi_{\alpha}^{-1}"'] 
                                              & & K(\beta)
  \end{tikzcd}
  \end{equation}
  Therefore taking $\overline{\phi}: F(\alpha) \to F(\beta)
  = \psi_{\beta}\phi'\psi_{\alpha}^{-1}$ gives us the extension we require, as
  we needed.
\end{proof}

This result similar to our previous is strong. It allows us to extend
isomorphisms between fields to isomorphisms between extensions by trancendental
numbers. 
