The next tool that we are going to need to use is transfinite induction.
Intuitively this is a way for us to extend the notion of usual induction on the
natural numbers and allows us to perform a similar process on using what are
known as ordinals.

Before we can talk about what ordinals are, it is important for us to introduce
some other concepts first. Suppose we have a set $X$ recall that a linear order
is a relation $<$ on $X$ satisfying the following properties:
\begin{enumerate}
  \item Irreflexive: $\neg(a < a)$
  \item Transitive: If $a < b$ and $b < c$ then $a < c$.
  \item Linearity: For all $x,y \in X$ either $x < y, x = y$ or $y < x$.
\end{enumerate}

We are going to focus on a specific type of linear order known as a well order.

\begin{defn}[Well-Orders]
  A linear order $(X, <)$ is said to be a well order if every $S \subseteq X$
  has a minimal element.
\end{defn}

If $(X, <_X)$ and $(Y, <_Y)$ are linear orders we say that a bijective map
$\phi: X \to Y$ is an isomorphism of linear orders if for all $x, y \in X$ we
have that if $x < y$ then $\phi(x) < \phi(y)$. We say that $(X, <_X), (Y, <_Y)$
are isomorphic if there is a linear order isomorphism between them.

Something that is worth noting is that one consequence of the axiom of choice
is known as the Well Ordering Principle. The well ordering principle states
that on every set there exists an ordering which is a well ordering. Notice
that neither $\mathbb{Q}$ nor $\mathbb{Z}$ are well orders in their usual
ordering as they don't have a least element. The well ordering principle tells
us that there exist orderings on both $\mathbb{Q}$ and $\mathbb{Z}$.\\

Next let look at the ``set theorists natural numbers''. One of the main things we
can try and do to motivate the definition of an ordinal is to turn everything
we care about working with into sets. The main goal of an ordinal is to
extend a sort of ``order'' into an (uncountbly) infinite setting. First lets
encode the natural numbers as sets.

\begin{align*}
  0 &= \{\} \\
  1 &= \{\{\}\} \\
  2 &= \{\{\}, \{\{\}\}\} \\
  3 &= \{\{\}, \{\{\}, \{\{\} \}\}\} 
.\end{align*}
Or we can re-write this as
\begin{align*}
  0 &= \{\} \\
  1 &= \{0\} \\
  2 &= \{0,1\} \\
  3 &= \{0,1,2\} 
.\end{align*}

Notice that these sets have the property that containment and inclusion are the
same thing. For example $2 \in 3$ and $2 \subseteq 3$ based on how $2$ and $3$ 
are defined.  This is what is known as a transitive set.

\begin{defn}[Transitive]
  A set $X$ is said to be transitive if whenever $x \in X$ then $x \subseteq
  X$.
\end{defn}

We are now able to give the defintion of an ordinal.

\begin{defn}[Ordinal]
  An ordinal is a transitive set $\alpha$ such that $(\alpha, \in)$ is
  a well-ordering.
\end{defn}


\begin{lemma}
  The class of ordinals is well ordered by inclusion. In particular if
  $\alpha, \beta$ are any two ordinals then either $\alpha < \beta$, $\beta
  < \alpha$ or $\alpha = \beta$. Here $<$ is being used interchangably with
  $\in$.
\end{lemma}

Ordinals, as the name implies, are designed to capture and extend the notion of
``order'' beyond just the natural numbers. Indeed if we take any well ordering
$(A, <)$ there is a unique ordinal $\alpha$ which is isomorphic to $A$. So any
well ordered set corresponds to a unique ordinal.

There are two types of ordinals, successor and limit. We say that $\alpha$ is
a successor ordinal if $\alpha = \beta + 1 = \beta \cup \{\beta\}$ 
for some ordinal $\beta$. We say that $\alpha$ is a limit ordinal 
if $\alpha = \bigcup_{\beta < \alpha} \beta$. To illustrate this consider the 
natural numbers $\omega$. Notice that each natural number is a successor 
ordinal for example $2 = 1 + 1$, $3 = 2 + 1$, and so on. However the $\omega$
itself is a limit ordinal as $\omega = \bigcup_{n \in \mathbb{N}} n$.

\begin{lemma}
  If $\alpha$ is an ordinal then $\alpha + 1 = \alpha \cup \{\alpha\}$ is an
  ordinal and $\alpha < \alpha + 1$. If $\gamma$ is an ordinal and $A$ is
  a collection of ordinals such that $\alpha < \gamma$ for all $\alpha \in A$ 
  then $\beta = \bigcup_{\alpha \in A} \alpha$ is an ordinal and $\beta
  < \gamma$.
\end{lemma}

The details about how ordinals work can be found in \cite{Schimmerling2011-li}
however for our purposes we are simply going to assume the above
facts about ordinals. With these facts in mind we are now able to state the most 
important results from this section

\begin{thm}[Transfinite Induction]
  Suppose that $P(\alpha)$ is a statement in a variable $\alpha$. Assume that
  for every ordinal $\beta$ we have that
   \[
     (\forall \alpha < \beta)\  P(\alpha) \implies P(\beta)
  .\] 
  Then $P(\gamma)$ holds for every ordinal $\gamma$.
\end{thm}

A proof of this statement can be found in \cite{Schimmerling2011-li} but for our
purposes it suffices to notice that this allows us to prove things for all
ordinals. In practice, we break this up into cases in a similar fashion to what
we do for regular induction on the natural numbers.

\begin{enumerate}
  \item Does the statement hold for $0$
  \item Assume that $P(\alpha)$ is true, show that this implies  $P(\alpha + 1)$.
  \item Suppose that $\beta$ is a limit ordinal and $P(\alpha)$ holds for all
    $\alpha < \beta$ show that this implies $P(\beta)$.
\end{enumerate}

This effectively allows us to do induction on ordinals and allows us to perform
``induction style'' proofs in settings where that might not necessarily be
appropriate. We will see more examples of this in the next section.

The last tool that we are going to need from this section are cardinals. We
want to relate our usual notion of cardinality to the above discussion about
ordinals. Recall that two sets have the same cardinality if there is
a bijection between them. Now let $A$ be a set, by the well ordering principle
we can assign $A$ a well ordering $<$. We define the cardinality of $A$,
denoted $|A|$, to be the least ordinal $\alpha$ in bijection with $A$. The
importance of cardinals for our purposes is that they are going to allow us
``enumerate'' sets, similar to how one would enumerate $\mathbb{Z}$ or
$\mathbb{Q}$. Suppose that $|A| = \kappa$ then since $\kappa$ is an ordinal we
can enumerate $A$ as $\{a_{\alpha}: \alpha < \kappa\}$.

