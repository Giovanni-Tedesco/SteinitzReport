The next tool that we are going to need to use is transfinite induction.
Intuitively this is a way for us to extend the notion of usual induction on the
natural numbers and allows us to perform a similar process on using what are
known as ordinals.

Before we can talk about what ordinals are, it is important for us to introduce
some other concepts first.

\begin{defn}[Linear Ordering]
  A linear order is a set $X$ and a relation $< \subseteq X \times X$ that
  satisfies the following properites:
  \begin{enumerate}
    \item Irreflexive: $\neg a < a$
    \item Transitive: If $a < b$ and $b < c$ then $a < c$.
    \item Linearity: For all $x,y \in X$ either $x < y, x = y$ or $y < x$.
  \end{enumerate}
\end{defn}

\begin{defn}[Well-Orders]
  A linear order $(X, <)$ is said to be a well order if every $S \subseteq X$
  has a minimal element.
\end{defn}

Consider the two examples above. We see that while they are both certainly
linear orders, we see that the natural numbers are a well order as any subset
of the natural numbers has a minimal element. Notice that neither $\mathbb{Q}$
or $\mathbb{Z}$ with their usual ordering are well orders as neither
$\mathbb{Q}$ nor $\mathbb{Z}$ has a minimal element.

Something that is worth mentioning is that by the Axiom of Choice we can
place a well order on every set. The above examples of $\mathbb{Q}$ and
$\mathbb{Z}$ have well orders, but that does not mean that they are well
ordered in their usual order.

Let's look at the ``set theorists natural numbers''. One of the main things we
can try and do to motivate the definition of an ordinal is to turn everything
we care about working with into sets. The main goal of an ordinal is to
extend a sort of ``order'' into an (uncountbly) infinite setting. First lets
encode the natural numbers as sets.

\begin{align*}
  0 &= \{\} \\
  1 &= \{\{\}\} \\
  2 &= \{\{\}, \{\{\}\}\} \\
  3 &= \{\{\}, \{\{\}, \{\{\} \}\}\} 
.\end{align*}
Or we can re-write this as
\begin{align*}
  0 &= \{\} \\
  1 &= \{0\} \\
  2 &= \{0,1\} \\
  3 &= \{0,1,2\} 
.\end{align*}

We can continue this process onwards and it gives us the natural numbers
$\omega = \{1, 2, 3, \ldots\}$. But where do we go from here? Well naturally we
can continue on doing what we were doing. Notice that in the previous process
$n + 1 = n \cup \{n\}$. We are going to do the exact same thing now. We take
$\omega + 1 = \omega \cup \{\omega\}$. We can again do this infinitely and that
gives us $\omega + \omega$. Repeating a similar process for multiplication and
exponentiation and onwards allows us to continue this process further and
further. Notice that at each step we are doing one of 2 things. We are either
adding a new set, for example in the case of $\omega + 1$, or we are taking the
union of many sets, for example $\omega = \bigcup_{n \in \mathbb{N}} n$. Going
forward these are going to be known respectively as the successor case, and the
limit case. 

Stepping back for a moment we look at the ``set theorists natural numbers''
again. Notice here that $2 \in 3$ by the definition of the set but we also have
that $1 \subseteq 3$ when we look at their definition as sets. This is true
beyond just the naturals. Notice that $\omega + 1 = \omega \cup \{\omega\}$.
Thus not only is $\omega \in \omega + 1$ but $\omega \subseteq \omega + 1$.

\begin{defn}[Transitive]
  A set $X$ is said to be transitive if whenever $x \in X$ then $x \subseteq
  X$.
\end{defn}

Notice that the sets we defined above are all transitive. For example $2 \in 3$
and $2 \subseteq 3$ based on how $2$ and $3$ are defined. 

\begin{defn}[Ordinal]
  An ordinal is a transitive set $\alpha$ such that $(\alpha, \in)$ is
  a well-ordering.
\end{defn}

This in particular tells us that $\alpha = \{\beta: \beta < \alpha\}$. 
There are two main types of ordinals. Successor ordinals, and limit ordinals.

\begin{defn}[Successor and Limit Ordinals]
  We say that an ordinal $\beta$ is a successor ordinal if $\beta = \alpha + 1$ 
  for some ordinal $\alpha$. Ordinals which are not successor ordinals are
  called limit ordinals.
\end{defn}

The details about how ordinals work can be found in \cite{Schimmerling2011-li} however for our purposes we are going to assume the following
facts about ordinals.

\begin{lemma}
  The class of ordinals is well ordered by inclusion. In particular if
  $\alpha, \beta$ are any two ordinals then either $\alpha < \beta$, $\beta
  < \alpha$ or $\alpha = \beta$. Here $<$ is being used interchangably with
  $\in$.
\end{lemma}

\begin{lemma}
  If $\alpha$ is an ordinal then $\alpha + 1 = \alpha \cup \{\alpha\}$ is an
  ordinal and $\alpha < \alpha + 1$. If $\gamma$ is an ordinal and $A$ is
  a collection of ordinals such that $\alpha < \gamma$ for all $\alpha \in A$ 
  then $\beta = \bigcup_{\alpha \in A} \alpha$ is an ordinal and $\beta
  < \gamma$.
\end{lemma}

With these facts in mind we are now able to state the most important results
from this section

\begin{thm}[Transfinite Induction]
  Suppose that $P(\alpha)$ is a statement in a variable $\alpha$. Assume that
  for every ordinal $\beta$ we have that
   \[
     (\forall \alpha < \beta)\  P(\alpha) \implies P(\beta)
  .\] 
  Then $P(\gamma)$ holds for every ordinal $\gamma$.
\end{thm}

A proof of this statement can be found in \cite{Schimmerling2011-li} but for our
purposes it suffices to notice that this allows us to prove things for all
ordinals. In practice, we break this up into cases in a similar fashion to what
we do for regular induction on the natural numbers.

\begin{enumerate}
  \item Does the statement hold for $0$
  \item Assume that $P(\alpha)$ is true, show that this implies  $P(\alpha + 1)$.
  \item Suppose that $\beta$ is a limit ordinal and $P(\alpha)$ holds for all
    $\alpha < \beta$ show that this implies $P(\beta)$.
\end{enumerate}

This effectively allows us to do induction on ordinals and allows us to perform
``induction style'' proofs in settings where that might not necessarily be
appropriate. We will see more examples of this in the next section.

The last tool that we are going to need from this section are Cardinals. We
have an intuitive notion for what it means for finite sets to be of the same
size. In particular this is when they have the same number of elements. For
infinite sets we need a better criteria for when they are the ``same size''.
Suppose that $(A, <)$ is any well order. By a result in
\cite{Schimmerling2011-li} we know that there is a least ordinal $\alpha$ 
isomorphic to $(A, <)$. We define $|A| = \alpha$.

\begin{defn}[Cardinal]
  We say that an ordinal $\alpha$ is a cardinal if $|\alpha| = \alpha$.
\end{defn}

We see that $\omega$ is a cardinal, we can define $\omega_1 = \{\alpha:
|\alpha| = \omega\}$. In a similar fashion we can define $\omega_{2}
= \{\alpha: |\alpha| = \omega_{1}\}$ and so on. Notice that $\omega_1$ is
uncountable as if $\omega_1$ was countable then $\omega_1 \in \omega_1$ which
is not possible. Similarly for the case of $\omega_2$ and so on. This means
that each $\omega_{i}$ has strictly larger cardinality than the previous. In
this fashion we can describe the cardinal $\omega_{\alpha}$ for an ordinal
$\alpha$ using transfinite induction.
\begin{enumerate}
  \item The base case is $\omega_0 = \omega$
  \item For the successor case suppose that we have defined $\omega_{\alpha}$ 
    then we can define $\omega_{\alpha + 1} := \{\beta: |\beta| = \alpha\} $
  \item If $\alpha$ is a limit ordinal we can define $\omega_{\alpha}
    = \sup_{\beta < \alpha} \omega_{\alpha}$
\end{enumerate}
Since ordinals and cardinals are fundamentally different things, we want to make that
distinction. When talking about the ordinal $\omega_{\alpha}$ we will write it
as such. When we are talking about the cardinal $\omega_{\alpha}$ we write it
as $\aleph_{\alpha}$. The important thing from this section to know is that if
$A$ is a set and $|A| = \kappa$ then we can ``enumerate'' this set by
considering the ordinal $\beta$ corresponding to $\kappa$ and writing $A
= \{a_{\alpha}: \alpha < \beta\}$. This is something that we already do for
countable sets. For example we can enumerate $\mathbb{Q}$ by $\{q_1, q_2,
\ldots, q_{n}\}$, we are now just extending that notion to an infinite setting.
