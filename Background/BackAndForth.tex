The actual technique that we are going to use in the proof of Steinitz's
Theorem is going to be a combination of transfinite induction, and an analogue
of what is known as a back and forth argument.

A back and forth argument is a useful tool in constructing isomorphisms between
two structures. In order to talk about these we need to introduce the notion of
a partial embedding.

\begin{defn}[Partial Embedding]
  Suppose that $\mathcal{M}$ and $\mathcal{N}$ are $\mathcal{L}$ structures.
  Suppose that $A \subseteq M$ and $B \subseteq N$. We say that $f: A \to B$ is
  a partial emebedding if $f$
  preserves the relations and functions of $\mathcal{L}$ 
\end{defn}

The general goal now is to build a sequence of partial embeddings $f_{0}
\subseteq f_{1} \subseteq f_{2} \subseteq \ldots$ such that $f_{i}: A_{i}
\to B_{i}$ with $A_{i} \subseteq A_{i + 1}$ and $B_{i} \subseteq B_{i + 1}$
with the additional property that $\bigcup_{i \in \mathbb{N}} A_n = M,
\bigcup_{i \in \mathbb{N}} B_{i} = N$. Once all of these maps have been defined
we will be able to define our isomorphism $f: M \to N$ by simply choosing an
appropriate $A_{n}$ that our input lies in. The question now becomes how do we
do this?

The above process can be formulated in many ways but we are going to be using
games. Suppose that there are two players,
at the $i$-th stage of the game, the first player will ``play'' either an
element $m_{i} \in M$ or an element $n_{i} \in N$ and the goal of player 2 is
to place this element in the domain by playing an element $n_{i}$ or $m_{i}$
respectively. Going back and forth in this way gives a sequence of partial
embeddings $f_{i}$ and sets $A_{i}$ and $B_{i}$ satisfying the conditions that
we outlined above. Player 2 wins the game if this process can be continued
indefinitely thus resulting in a map $f: M \to N$. 

%\begin{ex}
%We can now give a much weakened proof of Steinitz's Theorem. Suppose that $F$ 
%and $K$ are countable algebraically closed fields (of characteristic 0). 
%We construct our partial isomorphism as follows. Since we know that both $F$ 
%and $K$ contain $\mathbb{Q}$ as a subfield we let $f_{0}: \mathbb{Q} \to
%\mathbb{Q}$ be our first partial isomorphism. Next we let $M = \{\alpha_{i}
%\in F: \alpha_{i} \not\in \mathbb{Q}\}$ and $N = \{\beta_{i} \in N: \beta_{i}
%\not\in \mathbb{Q}\}$ be an enumeration of $M$ and $N$ respectively. 
%Now suppose that on turn 1 Player 1 plays $\alpha_{1} \in M$ (we take this
%without loss of generality, as the case for $\beta_{1} \in N$ is the same)
%then by Proposition \ref{propn:algextend} there is a $\beta_{i} \in N$ such that
%$f_{0}$ can be lifted to an isomorphism $f_{1}: \mathbb{Q}(\alpha_{1}) \to
%\mathbb{Q}(\beta_{i}))$. So we can take $A_{1} = \mathbb{Q}(\alpha_{1})$ and
%$B_{1} = \mathbb{Q}(\beta_{i})$. At the $i$-th turn of the game we can do this
%exact same process again by applying Proposition \ref{propn:algextend}
%regardless of what the first player plays. This gives us a winning strategy for
%the second player and gives us our result as we needed.\\ 
%\end{ex}

As we will come to see, the uncountable case is more difficult. The issue now
is that we need to keep in mind that there is a limit case we need to account
for. Suppose that  $\mathcal{M}$, $\mathcal{N}$ are two (uncountably) infinite
$\mathcal{L}$ structures such that $|M| = |N| = \kappa$. Our goal is going to be to 
construct sets
\[
  A_0 \subseteq A_1 \subseteq \ldots \subseteq A_{\beta} \subseteq M
\] 
and
\[
B_0 \subseteq B_1 \subseteq \ldots \subseteq B_{\beta} \subseteq N
\] 
and a sequence of partial isomorphisms $f_0 \subseteq f_1 \subseteq \ldots
f_{\beta} \subseteq \ldots$. With these we would have a full isomorphism  $f:
\bigcup_{\beta < \kappa} A_{\beta} \to \bigcup_{\beta < \kappa} B_{\beta}$ 
where $f = \bigcup_{\beta < \kappa} f_{\beta}$ will be our full isomorphism. We
notice that this is the exact same thing that we want from above. It is more
difficult to formulate this in terms of games. However coming up with such
a construction suffices for us to prove that these two sets are isomorphic.


